%%%%%%%%%%%%%%%%%%%%%%%%%%%%%%%%%%%%%%%%%
% Medium Length Professional CV
% LaTeX Template
% Version 2.0 (8/5/13)
%
% This template has been downloaded from:
% http://www.LaTeXTemplates.com
%
% Original author:
% Trey Hunner (http://www.treyhunner.com/)
%
% Important note:
% This template requires the resume.cls file to be in the same directory as the
% .tex file. The resume.cls file provides the resume style used for structuring the
% document.
%
%%%%%%%%%%%%%%%%%%%%%%%%%%%%%%%%%%%%%%%%%

%----------------------------------------------------------------------------------------
%	PACKAGES AND OTHER DOCUMENT CONFIGURATIONS
%----------------------------------------------------------------------------------------

\documentclass{resume} % Use the custom resume.cls style

\usepackage[left=0.75in,top=0.6in,right=0.75in,bottom=0.6in]{geometry} % Document margins

\name{Kristen Masada} % Your name
\address{1 Greene St. \\ Athens, OH 45701} % Your address
\address{km942412@ohio.edu} % Your email

\begin{document}

%----------------------------------------------------------------------------------------
%	EDUCATION SECTION
%----------------------------------------------------------------------------------------

\begin{rSection}{Education}

{\bf Ohio University} \hfill {\em 2020} \\ 
M.S. in Computer Science \\

{\bf Ohio University} \hfill {\em 2018} \\ 
B.S. Honors Tutorial College Computer Science \smallskip \\
Thesis: Chord Recognition in Symbolic Music: A Segmental CRF Model, Segment-Level Features, and Comparative Evaluations on Classical and Popular Music \\
Overall GPA: 4.0

\end{rSection}

%----------------------------------------------------------------------------------------
%	WORK EXPERIENCE SECTION
%----------------------------------------------------------------------------------------

\begin{rSection}{Experience}

\begin{rSubsection}{Ohio University Honors Tutorial College}{May 2017 - Aug. 2017}{Research Apprentice}{Athens, OH}
\item Worked under Dr. Razvan Bunescu, testing the automatic chord recognition system that we created on a variety of datasets and preparing our conference paper (see ‘Publications’ section below) for final submission.
\end{rSubsection}

%------------------------------------------------

\end{rSection}

%----------------------------------------------------------------------------------------
%	PUBLICATIONS SECTION
%----------------------------------------------------------------------------------------

\begin{rSection}{Publications}
\item K. Masada and R. Bunescu, ``Chord recognition in symbolic music: a segmental CRF model, segment-level features, and comparative evaluations on classical and popular music," {\em Trans. of the Inter. Soc. for Music Information Retrieval}, vol. 2, no. 1, pp. 1-13, Jan. 2019. [Online]. Available: https://transactions.ismir.net/articles/10.5334/tismir.18
\item K. Masada and R. Bunescu, ``Chord recognition in symbolic music using semi-Markov Conditional Random Fields," in {\em Proc. 18th Inter. Soc. for Music Information Retrieval Conf.}, Suzhou, China, Oct. 2017. [Online]. Available: https://kristenmasada.github.io/publications/ismir17/ismir17.pdf
\end{rSection}

%----------------------------------------------------------------------------------------
%	HONORS SECTION
%----------------------------------------------------------------------------------------
% Reformat this section!
\begin{rSection}{Honors}
\item 1st Place Research Poster  \hfill {\em 2019} \\
{\em Ohio Celebration of Women in Computing Conference }
\item Outstanding Senior in Computer Science  \hfill {\em 2018} \\
{\em Russ College of Engineering and Technology }
\item 1st Place Research Poster, EECS-1 Session  \hfill {\em 2018} \\
{\em Ohio University Student Research and Creative Activity Expo }
\item Student Author Grant \hfill {\em 2017} \\
{\em International Society of Music Information Retrieval }

%----------------------------------------------------------------------------------------
%	TECHNICAL STRENGTHS SECTION
%----------------------------------------------------------------------------------------

%\begin{rSection}{Technical Strengths}

%\begin{tabular}{ @{} >{\bfseries}l @{\hspace{6ex}} l }
%Computer Languages & Prolog, Haskell, AWK, Erlang, Scheme, ML \\
%Protocols \& APIs & XML, JSON, SOAP, REST \\
%Databases & MySQL, PostgreSQL, Microsoft SQL \\
%Tools & SVN, Vim, Emacs
%\end{tabular}

%\end{rSection}

%----------------------------------------------------------------------------------------
%	EXAMPLE SECTION
%----------------------------------------------------------------------------------------

%\begin{rSection}{Section Name}

%Section content\ldots

%\end{rSection}

%----------------------------------------------------------------------------------------

\end{document}

